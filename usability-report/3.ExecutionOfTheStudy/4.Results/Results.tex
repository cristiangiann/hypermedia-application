\documentclass[../../UsabilityReport.tex]{subfiles}
\begin{document}
\section{Results}
	After having collected the data, they have been analyzed in order to evaluate the website through the variables defined in Chapter \ref{2.3}. \\
	
	Every user completed all the tasks, so the calculated Effectiveness (\textit{V1}) is 100\%. The expectations were high due to the simple structure of the website, and this result confirmed it. About Efficiency (\textit{V2}), all the tasks have an acceptable average time. Scores that most deviate from the average are generally due to difficulties to locate specific elements, like the \textit{Past Events} section on the \textit{Event} page in a couple of cases, but in few seconds everyone succeded autonomously in their tasks. With respect to the order by which tasks have been assigned, the last two took less time to be completed, compared with their average, than the first two, giving enough certainty to state that the user, after completing her first two tasks (and in some cases right after the first one), was yet confident with the website and its navigation. This assertion is also confirmed by the survey already cited in \ref{Execution-paragraph}, whose answers are reported in the \textit{Table 3.3}\\
	Only an Error (\textit{V3}) of wrong path occurred during a task (i.e. a user tried to find the address of \textit{Lemon Peel} office in the \textit{Association} page instead of the \textit{Contacts} section), and two wrong actions in the \textit{Task 4}, while finding the instruments of a specific type, probably due to the low visibility of the filters in the \textit{Musical Instrument} page. \\
	
	The survey shows good scores about the statements presented to the participants of the test; the highest ones are related to the easiness of the website and how fast information can be retrieved. The lowest score, even if it is a good score overall, is related to the engagement of the home page. This might be related to the interests of the sample of users that might not match the typical user of \textit{Lemon Peel}. It is suggested to investigate this aspect further in order to have a better understanding of the causes. \\
	From the survey, what is noticeable is that the users felt no Wandering periods (\textit{V6}) or Disorientation (\textit{V5}) during the navigation (\textit{S1}, \textit{S2}, \textit{S3}) and that the website gives Confidence (\textit{V7}) to the users with its graphical design and structure, so that they could find the information they needed without feeling any inconsistency between pages (\textit{S6}, \textit{S7}, \textit{S9}, \textit{S10}). We measured also the users' Satisfaction (\textit{V4}), whose results are positive showing there were limited or no difficulties during the navigation (\textit{S4}, \textit{S8}), that is also confirmed by \textit{S5} about the lack of disorientation.
	
	\begin{table}[htb]
		\centering
		\caption{Aggregated results of the usability survey}
		\begin{tabular}{|c|c|c|c|c|c|}
			\hline
			& \multicolumn{1}{|p{14mm}|}{Strongly Disagree} & \multicolumn{1}{|p{14mm}|}{Disagree} & \multicolumn{1}{|p{14mm}|}{Neutral} & \multicolumn{1}{|p{14mm}|}{Agree} & \multicolumn{1}{|p{14mm}|}{Strongly Agree} \\ \hline
			\textit{S1}  & & & & \multicolumn{1}{c|}{1} & \multicolumn{1}{c|}{5} \\ \hline
			\textit{S2}  & \multicolumn{1}{c|}{2} & \multicolumn{1}{c|}{4} & & & \\ \hline
			\textit{S3}  & & & & \multicolumn{1}{c|}{3} & \multicolumn{1}{c|}{3} \\ \hline
			\textit{S4}  & \multicolumn{1}{c|}{3} & \multicolumn{1}{c|}{3} & & & \\ \hline
			\textit{S5}  & & & & \multicolumn{1}{c|}{1} & \multicolumn{1}{c|}{5} \\ \hline
			\textit{S6}  & & & & \multicolumn{1}{c|}{2} & \multicolumn{1}{c|}{4} \\ \hline
			\textit{S7}  & & & & \multicolumn{1}{c|}{2} & \multicolumn{1}{c|}{4} \\ \hline
			\textit{S8}  & & & & \multicolumn{1}{c|}{2} & \multicolumn{1}{c|}{4} \\ \hline
			\textit{S9}  & \multicolumn{1}{c|}{3} & \multicolumn{1}{c|}{3} & & & \\ \hline
			\textit{S10} & & & \multicolumn{1}{c|}{1} & \multicolumn{1}{c|}{2} & \multicolumn{1}{c|}{3} \\ \hline
		\end{tabular}
	\end{table}
\end{document}
