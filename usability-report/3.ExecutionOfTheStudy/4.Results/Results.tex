\documentclass[../../UsabilityReport.tex]{subfiles}
\begin{document}
\section{Results}
	After having collected the data we analyzed them in order to evaluate the website through the variables defined in Chapter \ref{2.3}. \\
	
	With an Effectiveness of 100\%, the test proved to have an Efficiency of 2 minutes and 51 seconds, while only an Error of wrong path occurred during a task (i.e. a user tried to find the address of \textit{Lemon Peel} office in the \textit{Association} page instead of the \textit{Contacts} section). \\
	
	One thing we noticed is that, with respect to the order by which tasks have been assigned, the last two tasks took less time to be completed than the first two, giving us enough certainty to state that the user, after completing its first two tasks (and in some cases right after the first one), was yet confident with the website and its navigation. \\
	This assertion is also confirmed by the survey already cited in \ref{Execution-paragraph}, whose answers are reported in the table below. What is noticeable is that there were no wandering periods or disorientation among users during the test (S1 to S3)  and the website is well designed and structured so that every user could find detailed information without feeling any inconsistency between pages (S6-S7-S9-S10). We measured also users' satisfaction (S4-S8), whose results are positives showing there were no difficulties during the navigation, that is also confirmed by S5 about the lack of disorientation.
	
	\begin{table}[htb]
		\resizebox{\textwidth}{!}{%
			\begin{tabular}{|l|c|c|c|c|c|}
				\hline
				& \multicolumn{1}{l|}{Strongly Agree} & \multicolumn{1}{l|}{Agree} & \multicolumn{1}{l|}{Neutral} & \multicolumn{1}{l|}{Disagree} & \multicolumn{1}{l|}{Strongly Disagree} \\ \hline
				S1  & 5                                   & 2                          &                              &                               &                                        \\ \hline
				S2  &                                     &                            &                              & 4                             & 2                                      \\ \hline
				S3  & 3                                   & 3                          &                              &                               &                                        \\ \hline
				S4  &                                     &                            &                              & 3                             & 3                                      \\ \hline
				S5  & 5                                   & 1                          &                              &                               &                                        \\ \hline
				S6  & 4                                   & 2                          &                              &                               &                                        \\ \hline
				S7  & 4                                   & 2                          &                              &                               &                                        \\ \hline
				S8  & 4                                   & 2                          &                              &                               &                                        \\ \hline
				S9  &                                     &                            &                              & 3                             & 3                                      \\ \hline
				S10 & 3                                   & 2                          & 1                            &                               &                                        \\ \hline
			\end{tabular}%
		}
	\end{table}
\end{document}
