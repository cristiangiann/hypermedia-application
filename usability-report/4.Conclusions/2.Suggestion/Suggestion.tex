\documentclass[../../UsabilityReport.tex]{subfiles}
\begin{document}
\section{Suggestion for Improvements}
	The whole evaluation process gave some suggestion and ideas to improve the user experience while using the website, through the study of the user interaction during the tests and the through the comment section at the end of the survey.
	Most of the listed suggestions are quick in terms of time needed for the developmente, but some of them might significantly improve the first impact for new users that land on the website for the first time:
	\begin{itemize}
		\item
			Move the dropdown menus from the right to the left of the page. Some users did not immediately consider them when they visited the pages for the first time. Even if in all the tests the users always finded it autonomusly, someone might not see it 
		\item
			Expand the \textit{Past Events} section by default making their images black and white in the \textit{Events} page or make the \textit{Past} and \textit{Next} titles more visible. A past event can be useful to retrieve information about its organiser, so the suggestion is to show it a better way.
		\item
			The \textit{Musical Instrument} page contains information about its region and type. The instruments related to it contains the ones from the same region and of the same type together. In order to distinguish the ones from the others, two solutions can be implemented: the first one is creating two sections on the \textit{Musical Instrument} page that separate them; the second one is to create a link from the type and region sections that bring the user in the \textit{Musical Instruments} introductory page, automatically filtering by the selected information. The page will remain the same, but the dropdown menus will be set by a parameter taken from the url.
		\item
			Add a link from \textit{Association page} to \textit{Where we are} page.
		\item
			Highlight the hierarchy of the titles and text modifying the font weights or another font (Google Font suggest Oswald font to be used with the current one).
		\item
			Chromatically standardize the several pictures of the website and the palette used to offer the desired mood to the user.
	\end{itemize}
\end{document}
