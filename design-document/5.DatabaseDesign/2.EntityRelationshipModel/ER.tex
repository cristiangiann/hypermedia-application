\documentclass[../../DD.tex]{subfiles}
\begin{document}
\section{Entity-Relationship Model}

	Entity Relationship Model (ER Modeling) is a graphical approach to database design. It is a high-level data model that defines data elements and their relationship for a specified system. An ER model is used to represent real-world entities; entities are things or objects distinguishable from surrounding environment, and they can have relationships with each other.\\
	Cardinality constraints define the minimum and maximum number of relationships in which an entity can participate; in the following diagram, the Min-Max / ISO notation has been used, in which the cardinality values for an entity are positioned near that entity (e.g. a Musical Instrument entity participates in exactly one relationship with an Italian Region entity; the latter can have zero or more relationships with the Musical Instrument entity).
	\image{\textheight}{Database/ER.jpg}{Entity-Relationship model}{er-model}	
\end{document}
