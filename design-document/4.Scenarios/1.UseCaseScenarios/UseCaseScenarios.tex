\documentclass[../../DD.tex]{subfiles}
\begin{document}
	\section{Use case scenarios}
		Use case scenarios help to understand how typical users are going to use the application, assuming their profile and the needs the application is going to solve, and highlighting how these standard users interact with the website.

		\subsection{Use case 1}
		    Giorgio is a 80 years old man from South Italy, with a passion for technology and old musical instruments. He has heard of an association in his town whose aim is to teach how to play Italian musical instruments of the past but also anything related to popular and folk music of 19th and 20th century and he's interested in learning more about it.\\
			Accessing the \textit{Lemon Peel} website, he clicks the \textit{Next Event} link from the \textit{Home Page} to check when he might learn more about the Association in person. From the description he notices that the only purpose of the \textit{Event} is to present the new courses, but he would like to know if there will be someone who play some old instruments. So, he clicks on the \textit{Event organizer} link, where he finds her contacts to ask more information about it.
			
			\image{\textheight}{scenarios/scen1-1.png}{Home Page}{scen1-1}
			\image{8 cm}{scenarios/scen1-2.png}{Event Page}{scen1-2}
			\image{8 cm}{scenarios/scen1-3.png}{Event organiser Page}{scen1-3}

		\newpage
		\subsection{Use case 2}
			Amanda is a 17 years old girl who founds a vintage instrument in her house's basement and she wants to learn more about it. She knows that there is an association whose aim is to teach the history of traditional Italian music and instruments.\\
			She opens the website and clicks on the \textit{Musical Instruments} link in the navigation bar; then, she finds that the ocarina, by its picture, is very similar to the instrument she has at home. From the \textit{Musical Instrument} page she reads the story behind it and she finds that there is a course to learn how to play this instrument, so she clicks on the \textit{Go to course} button to have more information about the day and time of each lesson to participate.
			
			\image{\textheight}{scenarios/scen2-1.png}{Home Page}{scen1-1}
			\image{8 cm}{scenarios/scen2-2.png}{Musical Instruments Page}{scen1-1}
			\image{8 cm}{scenarios/scen2-3.png}{Musical Instrument Page}{scen1-1}
			\image{8 cm}{scenarios/scen2-4.png}{Course Page}{scen1-1}

		\newpage
		\subsection{Use case 3}
			Ippolito is a middle-age man who owns a large collection of historical musical instruments of the past. He knows how to play many of them and he would like to teach it to people who are interested in traditional music.\\
			He goes on the \textit{Lemon Peel} association website and clicks on the \textit{Courses} landmark. He notices that in the page there is no reference to Baghèt courses, a bagpipe from Bergamo that he knows very well. Aiming to teach how to play it, he clicks on \textit{Contacts} landmark to ask to the association if they might be interested to offer this new \textit{Course}. Then, he writes a message explaining what he could bring to the association and clicks on \textit{Submit} button.
			
			\image{\textheight}{scenarios/scen3-1.png}{Home Page}{scen1-1}
			\image{8 cm}{scenarios/scen3-2.png}{Courses Page}{scen1-1}
			\image{8 cm}{scenarios/scen3-3.png}{Contact Page}{scen1-1}
\end{document}