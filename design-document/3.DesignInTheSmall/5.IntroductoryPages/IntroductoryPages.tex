\documentclass[../../DD.tex]{subfiles}
\begin{document}
\section{Introductory Pages \label{sect:2.1}}
	\subsection{Musical Instruments}
		\textit{Musical Instruments} introductory page contains a list of traditional \textit{Musical Instruments} that are interesting from the association's point of view. Each \textit{Musical Instruments} is represented by its name and a picture, and both of them are links to the instrument page. Two filters allow to show to the user a subset of these instruments: the first one filters them by type (e.g. Flutes, Aerophone...), the second one selects them by the region they come from.
		\newline
		\image{\textheight}{Wireframes/IntrMusicalInstruments.png}{Musical Instruments page wireframe}{musical-instruments-wireframe}
		\image{\textheight}{screenshots/MusicalInstruments.png}{Musical Instruments page screenshot}{musical-instruments-screenshot}

	\subsection{Courses}
		This introductory page contains a list of the courses hold by \textit{Lemon Peel} association. Each \textit{Course} has an image and a title that bring to the single \textit{Course} page, to read more information about it.
		\newline
		\image{\textheight}{Wireframes/IntrCourses.png}{Courses page wireframe}{courses-wireframe}
		\image{\textheight}{screenshots/Courses.png}{Courses page screenshot}{courses-screenshot}

	\subsection{Events}
		\textit{Events} page is divided into two main subsections containing next and past \textit{Events}. In this page, all \textit{Events} are represented by an image, a title and a short description that gives the idea of what the event is about. From this page it is possible to open the single-\textit{Event} pages to have more details about them. Through a filter, the user can select the \textit{Events} of a specific month, to have an immediate view of what the association is organizing in a defined time section.
		\newline
		\image{\textheight}{Wireframes/IntrEvents.png}{Events page wireframe}{events-wireframe}
		\image{\textheight}{screenshots/Events.png}{Events page screenshot}{events-screenshot}

	\end{document}
